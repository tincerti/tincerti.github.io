---
layout: splash
title: "Research"
permalink: /research/
author_profile: false
---

## Research overview

My research follows two related streams in comparative political economy. The main stream of my research is extraction of private gain from politics and public office. I have examined numerous topics within this stream, such as electoral punishment of corruption, regulatory capture by firms and interest groups, and award of contracts to firms with political connections. A second stream looks at economic drivers of comparative political behavior. Among other things, this line of research has examined how to encourage collective action to overcome regulatory capture. My research also explores quantitative research methods. I am particularly interested in the reliability and validity of measurement strategies, as well as improving accessibility to applied researchers and practitioners. 

I typically use field experiments or original data collection combined with observational econometrics to measure, e.g., effectiveness of strategies to reduce capture of public office or levels of private gain received. I have a regional specialization in Japan and East Asia, but examine these research areas across countries wherever they are salient.

Examples of publications and working papers in each of those domains can be found below. 

### Private gain from politics and public office

- Incerti, Trevor. ''[Corruption information and vote share: A meta-analysis and lessons for experimental design](https://www.cambridge.org/core/services/aop-cambridge-core/content/view/AB2ACE468B04EAB85CAF7379F9DF4817/S000305542000012Xa.pdf/corruption_information_and_vote_share_a_metaanalysis_and_lessons_for_experimental_design.pdf).'' *American Political Science Review*, 114.3 (2020): 761-774. [Online appendix](https://static.cambridge.org/content/id/urn:cambridge.org:id:article:S000305542000012X/resource/name/S000305542000012Xsupp001.pdf). [Replication data](https://dataverse.harvard.edu/dataset.xhtml?persistentId=doi:10.7910/DVN/HD7UUU).  
  <details><summary>Abstract</summary><p> Debate persists on whether voters hold politicians accountable for corruption. Numerous experiments have examined whether informing voters about corrupt acts of politicians decreases their vote share. Meta-analysis demonstrates that corrupt candidates are punished by zero percentage points across field experiments, but approximately 32 points in survey experiments. I argue this discrepancy arises due to methodological differences. Small effects in field experiments may stem partially from weak treatments and noncompliance, and large effects in survey experiments are likely from social desirability bias and the lower and hypothetical nature of costs. Conjoint experiments introduce hypothetical costly trade-offs, but it may be best to interpret results in terms of realistic sets of characteristics rather than marginal effects of particular characteristics. These results suggest that survey experiments may provide point estimates that are not representative of real-world voting behavior. However, field experimental estimates may also not recover the “true” effects due to design decisions and limitations. </p></details>

- Incerti, Devin, and Trevor Incerti. ''[Are regime changes always bad economics? Evidence from daily financial data](http://tincerti.github.io/files/regime_changes.pdf).'' (*under review*)
  <details><summary>Abstract</summary><p> Political instability is commonly thought to discourage investment and reduce economic growth. By contrast, we find that different types of “irregular” regime changes - coups, assassinations, or resignations - have disparate effects on stock returns. We examine daily returns of national stock indices in every country that experienced an irregular regime change subject to data availability. Using an event study approach, we show that abnormal returns following resignations are large and positive (4%), while those following assassinations are negative and smaller in magnitude (2%). The impact of coups tends to be negative (2%), but some events result in positive abnormal returns of 10% or more. Volatility increases during times of protest preceding resignations, but no clear directionality is present. We therefore find that the expected direction and magnitude of abnormal returns is dependent on the type of political event and its expected impact on economic policy. </p></details>
   
- Incerti, Trevor, and Hikaru Yamagishi. ''Do firms benefit from the revolving door? Evidence from Japan.''
   <details><summary>Description</summary><p> A growing literature finds high returns to firms connected to legislative office. Less attention has been paid to benefits from bureaucratic connections, despite well-documented bureaucratic revolving door hiring practices. Leveraging a 2009 law requiring Japanese bureaucratic agencies to report private sector hires of former civil servants, we construct a comprehensive dataset of all revolving door hires in Japan. Using this dataset and data on Japanese government contracts and loans, we test for systematic benefits that accrue to firms who hire former bureaucrats. </p></details>
 
- Incerti, Trevor, Sayumi Miyano, Diana Stanescu, and Hikaru Yamagishi. ''*Amakudata*: A new dataset of revolving door hires.''
   <details><summary>Description</summary><p> Political economists have long speculated about the effects of connections between bureaucracies and the private sector. However, data tracing flows of civil servants from the bureaucracy to the private sector remains rare. This article presents a new dataset, Amakudata, which contains individual-level data of all Japanese bureaucrats retiring into positions outside of the bureaucracy from 2009 to 2019. </p></details>
  

### Economic drivers of comparative political behavior
  
  - Incerti, Trevor. ''[Combatting capture in local politics: Evidence from eight field experiments](https://www.trevorincerti.com/files/capture_in_local_politics.pdf).'' [Pre-analysis plan](http://tincerti.github.io/files/cc_preanalysis.pdf).  
   <details><summary>Abstract</summary><p> Understanding how to motivate individuals with long-term collective interest to engage in costly political behavior is an enduring question in political economy. While renters have an economic incentive to participate in local politics and encourage housing growth, their participation lags that of homeowners who yield immediate financial returns from participation. I conducted 8 field experiments in 8 cities to investigate how to motivate renters to comment at city council meetings. Outreach highlighting the costs of abstention caused public comments to increase by 1.4 percentage points amongst those who opened the emails, with those already engaged in local politics particularly responsive to treatment. Treatment-induced comments represented 8% of total comments and 46% of pro-housing comments across all treated meetings. Overall, the results suggest that increasing the perception that abstention is costly is an effective motivator of real-world political participation, and that outreach can change the representativeness of civic bodies where increases in accessibility alone do not. </p></details>
   

### Quantitative research methods
  
- Carlson, Jacob, Trevor Incerti, and P.M. Aronow. "[Dyadic clustering in international relations](https://arxiv.org/abs/2109.03774)." (*under review*)
   <details><summary>Abstract</summary><p>  Quantitative empirical inquiry in international relations often relies on dyadic data. Standard analytic techniques do not account for the fact that dyads are not generally independent of one another. That is, when dyads share a constituent member (e.g., a common country), they may be statistically dependent, or "clustered." Recent work has developed dyadic clustering robust standard errors (DCRSEs) that account for this dependence. Using these DCRSEs, we reanalyzed all empirical articles published in International Organization between January 2014 and January 2020 that feature dyadic data. We find that published standard errors for key explanatory variables are, on average, approximately half as large as DCRSEs, suggesting that dyadic clustering is leading researchers to severely underestimate uncertainty. However, most (67% of) statistically significant findings remain statistically significant when using DCRSEs. We conclude that accounting for dyadic clustering is both important and feasible, and offer software in R and Stata to facilitate use of DCRSEs in future research. </p></details>
   
### Political economy of Japan

- Incerti, Trevor, Daniel Mattingly, Frances Rosenbluth, Seiki Tanaka, and Jiahua Yue. ''[Hawkish partisans: How political parties shape nationalist conflicts in China and Japan](https://www.cambridge.org/core/journals/british-journal-of-political-science/article/hawkish-partisans-how-political-parties-shape-nationalist-conflicts-in-china-and-japan/D625B68B3659A3CAD1A1D56E12AB45C3).'' *British Journal of Political Science*, 51.4 (2021): 1494-1515. [Online Appendix](https://static.cambridge.org/content/id/urn:cambridge.org:id:article:S0007123420000095/resource/name/S0007123420000095sup001.pdf). [Replication data](https://dataverse.harvard.edu/dataset.xhtml?persistentId=doi:10.7910/DVN/S4YXQB).  
  <details><summary>Abstract</summary><p> It is well known that regime types affect international conflicts. This article explores political parties as a mechanism through which they do so. Political parties operate in fundamentally different ways in democracies vs. non-democracies, which has consequences for foreign policy. Core supporters of a party in a democracy, if they are hawkish, may be more successful at demanding hawkish behavior from their party representatives than would be their counterparts in an autocracy. The study draws on evidence from paired experiments in democratic Japan and non-democratic China to show that supporters of the ruling party in Japan punish their leaders for discouraging nationalist protests, while ruling party insiders in China are less likely to do so. Under some circumstances, then, non-democratic regimes may be better able to rein in peace-threatening displays of nationalism. </p></details>

- Incerti, Trevor and Phillip Lipscy. ''[The politics of energy and climate change in Japan under Abe](http://tincerti.github.io/files/AS5804_01_Incerti_and_Lipscy.pdf).'' *Asian Survey*, 58.4 (2018).  
  <details><summary>Abstract</summary><p> Under what we call Abenergynomics, Japanese Prime Minister Abe Shinzo has used energy policy to support the growth objectives of Abenomics, even when the associated policies are publicly unpopular, opposed by utility companies, or harmful to the environment. We show how Abenergynomics has shaped Japanese policy on nuclear power, electricity deregulation, renewable energy, and climate change. </p></details> 

- Lipscy, Phillip Y., Kenji E. Kushida, and Trevor Incerti. ''[The Fukushima disaster and Japan's nuclear plant vulnerability in comparative perspective](https://pubs.acs.org/doi/pdfplus/10.1021/es4004813).'' *Environmental Science & Technology*, [47.12](http://tincerti.github.io/files/est_cover.jpg) (2013): 6082-6088.  \[Media coverage: [*The New Yorker*](https://www.newyorker.com/news/evan-osnos/sandy-fukushima-and-the-nuclear-industry)\]. 
  <details><summary>Abstract</summary><p> We consider the vulnerability of nuclear power plants to a disaster like the one that occurred at Fukushima Daiichi. Examination of Japanese nuclear plants affected by the earthquake and tsunami on March 11, 2011 shows that three variables were crucial at the early stages of the crisis: plant elevation, sea wall elevation, and location and status of backup generators. Higher elevations for these variables, or waterproof protection of backup generators, could have mitigated or prevented the disaster. We collected information on these variables, along with historical data on run-up heights, for 89 coastal nuclear power plants in the world. The data shows that 1. Japanese plants were relatively unprotected against potential inundation in international comparison, but there was considerable variation for power plants within and outside of Japan; 2. Older power plants and plants owned by the largest utility companies appear to have been particularly unprotected. </p></details>

- Incerti, Trevor and Phillip Lipscy. ''[The energy politics of Japan](https://www.oxfordhandbooks.com/view/10.1093/oxfordhb/9780190861360.001.0001/oxfordhb-9780190861360-e-21).'' In [*The Oxford Handbook of Energy Politics*](https://www.oxfordhandbooks.com/view/10.1093/oxfordhb/9780190861360.001.0001/oxfordhb-9780190861360) (2020). (*Peer reviewed*). 
  <details><summary>Abstract</summary><p> Japanese energy policy has attracted renewed attention since the 2011 Fukushima nuclear disaster. However, Japan’s energy challenges are nothing new; as a country poor in natural resources, it has long struggled to meet its energy needs. This chapter provides an overview of Japanese energy politics, focusing on three broad topics: Japan’s modernization and energy security challenges, the politics of the utilities sector and nuclear energy, and the politics of energy conservation and climate change. In addition, the chapter discusses factors specific to Japan, such as state-business relations in the utilities sector and institutional changes since the 1990s. Japan offers both compelling puzzles—several transformative shifts in energy conservation policy, limited emphasis on renewables despite persistent energy security concerns, and reinvigoration of nuclear energy despite the Fukushima disaster—as well as important empirical opportunities for theory testing. The chapter concludes by calling for additional research that integrates insights from Japan into broader theoretical and cross-national scholarship, examines Japanese energy policy within an international context, and uses rigorous causal identification strategies to evaluate Japanese energy policy. Finally, it identifies the politics of decarbonization in Japan as a critical area for future research. </p></details>
  
<br>

 <p float="left">
   <iframe width="95%" height="300px" scrolling="no" frameBorder="0" style="position:relative; top: 0px; left: 0px;" src="https://www.trevorincerti.com/files/abstract_wordcloud.html"></iframe>
</p>



